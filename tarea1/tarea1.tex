\documentclass[11pt,letterpaper]{article}
\usepackage[utf8]{inputenc}

%----- Configuración del estilo del documento------%
\usepackage{graphicx, fancyhdr, lastpage}
\usepackage[left=2cm,right=2cm,top=1.8cm,bottom=2.3cm]{geometry}

\pagestyle{fancy}
\fancyhf{}
\rfoot{\textit{Página \thepage \hspace{1pt} de \pageref{LastPage}}}

%------ Paquetes matemáticos básicos --------%
\usepackage{amsmath, amssymb, amsthm}

\begin{document}

%------ Encabezado -------- %
\begin{center}
  \begin{minipage}{3cm}
    \begin{center}
      \includegraphics[height=3.4cm]{unam_logo.png}
    \end{center}
  \end{minipage}\hfill
  \begin{minipage}{10cm}
    \begin{center}
      \textbf{\Large Universidad Nacional Autónoma de México}\\[0.2cm]
      \textbf{\large Facultad de Ciencias}\\[0.2cm]
      \textbf{Lógica Computacional | 2025-2}\\[0.4cm]
      \textbf{\Large Tarea 01}\\[0.1cm]
      \textbf{Docentes:}\\
      Noé Hernández \hspace{1em} Santiago Escamilla \hspace{1em} Ricardo López\\[0.3cm]
      \textbf{Autores:}\\
      Fernanda Ramírez Juárez \quad Ianluck Rojo Peña\\[0.3cm]
      \textbf{Fecha de entrega:} Martes 11 de febrero de 2025
    \end{center}
  \end{minipage}\hfill
  \begin{minipage}{3cm}
    \begin{center}
      \includegraphics[height=3.4cm]{fc_logo.png}
    \end{center}
  \end{minipage}
\end{center}

\bigskip
\hrule height 0.1pt
\bigskip

%------ Notas sobre la resolución --------%
\section*{Notas sobre la resolución}

\begin{quote}
  \textbf{Nota general:}  
  Los ejercicios fueron resueltos en base a las notas de clase (IcNota2.pdf) y a los comentarios dados en las sesiones del curso. Se tomaron en cuenta los siguientes puntos específicos:  
\end{quote}

\begin{itemize}
   \item \textbf{Ejercicios 1 y 2:} Se basan en las notas del profesor y los comentarios del 30 de enero y 6 de febrero.  
   \item \textbf{Ejercicio 4:} Resuelto con base en la sección '7. Conceptos semánticos básicos' en IcNota2.pdf y explicaciones del ayudante Santiago el 7 de febrero.  
   \item \textbf{Ejercicio 5:} Derivado de un ejercicio resuelto en clase el 30 de enero.  
\end{itemize}

\bigskip
\hrule height 0.1pt
\bigskip

%------ Contenido -------- %
\section*{Resolución de Ejercicios}

\begin{enumerate}
  
  % ---- Ejercicio 1 ----
  \item (1.5 pt.) Usando las siguientes claves:
    \begin{itemize}
       \item $p :=$ María está contenta.
       \item $q :=$ María pide una bicicleta por su cumpleaños.
       \item $r :=$ María recibe una bicicleta por su cumpleaños.
       \item $s :=$ María odia a Juan.
       \item $t :=$ Juan va a la playa.
       \item $u :=$ Juan está de vacaciones.
       \item $v :=$ El sol brilla.
    \end{itemize}
    
    Exprese en español las siguientes fórmulas llenando el cuadro que está abajo.
    \begin{enumerate}
       \item Siempre que María está contenta y el sol brilla, deja de odiar a Juan.
       \item Cuando brilla el sol, Juan va a la playa, si está de vacaciones.
       \item María está contenta siempre que Juan está de vacaciones y se va a la playa.
       \item Aunque María está contenta porque pidió una bicicleta para su cumpleaños y la ha recibido, odia a Juan.
       \item María recibirá una bicicleta en su cumpleaños sólo si la pide.
    \end{enumerate}
    
    \bigskip

    \begin{table}[t]
      \begin{center}
        \begin{tabular}{| c | c | c | c | c | c |}
          
          \hline 
          & $(u \land t) \rightarrow p$ & $\neg (r \land \neg q )$ & $(p \land v) \rightarrow \neg s$ & $(p \land (q \land r)) \land s$ & $v \rightarrow u \rightarrow t$ \\ \hline
          1 & & & \checkmark & & \\ \hline
          2 & & & & & \checkmark \\ \hline
          3 & \checkmark & & & & \\ \hline
          4 & & & & \checkmark & \\ \hline
          5 & & \checkmark & & & 
        \end{tabular}
      \end{center}
    \end{table}
    
    % ---- Ejercicio 2 ----
  \item (1 pt.) Desarrolle las siguientes sustituciones, además elimine los paréntesis que sean redundantes según el orden de precedencia de los operadores lógicos visto en clase:
    
    \begin{itemize}
    \item[\textbf{a})] $\left( \neg (p \land q) \leftrightarrow \left( (\neg q) \rightarrow (p \rightarrow s) \right) \right) \quad [p := (q \rightarrow s)] [s := (\neg p)]$
      
    \item[\textbf{b})] $\left( (p \lor q) \rightarrow ((\neg r) \leftrightarrow p) \right) \quad [r, p, q := p, q, r]$
    \end{itemize}
    
    \bigskip

  % ---- Ejercicio 3 ----
  \item (1 pt.) Tomando en cuenta la sintaxis para las fórmulas de la lógica proposicional definida en la Nota 01, reinserte tantos paréntesis como sea posible a la fórmula:
    \[
    (q \rightarrow p \rightarrow \neg r \land s) \lor \neg p
    \]

  \bigskip

  % ---- Ejercicio 4 ----
  \item (2 pts.) Sean $\Gamma$ y $\Delta$ dos conjuntos de oraciones de la lógica proposicional, y sean $\varphi$ y $\psi$ fórmulas de la lógica proposicional. Determine para cada una de las siguientes afirmaciones si es verdadera, con una demostración, o si es falsa, con un contraejemplo.

    \begin{itemize}
       \item Si $\Gamma \vdash \varphi \land \Delta \vdash \varphi$, entonces $\Gamma \cup \Delta \models \varphi$.
       \item Si $\Gamma \vdash \varphi$ y $\Delta \not\vdash \varphi$, entonces $\Gamma \cup \Delta \models \varphi$.
       \item Si $\Gamma \not\vdash \psi$, entonces $\Gamma \models \neg \psi$.
    \end{itemize}

  \bigskip

  % ---- Ejercicio 5 ----
  \item (1.5 pts.) Mediante interpretaciones decida si los siguientes conjuntos de proposiciones son satisfacibles:

    \begin{itemize}
       \item[a)] $\{ p \rightarrow q, (s \lor p) \land \neg q, \neg s \}$
       \item[b)] $\{ p \rightarrow q, q \leftrightarrow s, \neg p, \neg s \}$
    \end{itemize}

  \bigskip

  % ---- Ejercicio 6 ----
  \item (2 pts.) Usando deducción natural pruebe la validez de los siguientes:

    \begin{itemize}
       \item $p \rightarrow q, q \rightarrow r \lor s, \neg s, p \vdash r$
       \item $\neg p \lor q \vdash p \rightarrow q$
    \end{itemize}

  \bigskip

  % ---- Ejercicio 7 ----
  \item (2 pts.) Considere el siguiente argumento lógico:

    \textit{Si Sarah Connor destruye a Skynet en 1994, entonces no habrá Día del Juicio Final. Si no hay Día del Juicio Final, John Connor no enviará a su padre a 1984. Es condición necesaria que John Connor envíe a su padre a 1984, para que el mismo John nazca. Sarah Connor no destruye a Skynet en 1994, si John no nace. Por lo tanto, Sarah Connor no destruirá a Skynet en 1994.}

   Tradúzcalo a lógica proposicional y a través de tableaux semánticos determine si es correcto o no.

\end{enumerate}
\end{document}
